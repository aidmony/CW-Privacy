\NeedsTeXFormat{LaTeX2e}
\documentclass[11pt]{article}
\usepackage{url}
\usepackage{amsmath}
\usepackage{amsthm}
\usepackage{amssymb}
\usepackage{mathpartir}


\newcommand{\deftech}[1]{\textbf{#1}}
\newcommand\mrel{\mathop{\mathbf{r}}}
\newcommand\morel{\mathop{\mathbf{r}'}}
\newcommand\menv{\rho}
\newcommand\mval{v}
\newcommand\mans{a}
\newcommand\mint{i}
\newcommand\moint{j}
\newcommand\mbool{b}

\newcommand\plug[2]{#1[#2]}

\newcommand\merr{r}

\newcommand\mctx{\mathcal{C}}
\newcommand\mectx{\mathcal{E}}
\newtheorem{theorem}{Theorem}
\newcommand\Err{\mathit{Err}}
\newcommand\Plus{\mathit{Plus}}
\newcommand\Mult{\mathit{Mult}}
\newcommand\Succ{\mathit{Succ}}
\newcommand\Pred{\mathit{Pred}}
\newcommand\Eq{\mathit{Eq}}
\newcommand\True{\mathit{True}}
\newcommand\False{\mathit{False}}
\newcommand\If{\mathit{If}}
\newcommand\Div{\mathit{Div}}

\newcommand\reduce{\mathop{\mathbf{b}}}

\newcommand\areducename{\mathbf{a}}
\newcommand\areduce[2]{#1\;\areducename\;#2}

\newcommand\step{\rightarrow_\mathbf{b}}
\newcommand\multistep{\rightarrow^\star_\mathbf{b}}

\newcommand\astdstep{\longmapsto_{\reduce}}
\newcommand\astdmultistep{\longmapsto^\star_{\reduce}}

\newcommand\breducename{\mathbf{b}}
\newcommand\errreducename{\mathbf{err}}
\newcommand\propreducename{\mathbf{prop}}
\newcommand\bvreducename{\mathbf{bv}}
\newcommand\bstepname{\rightarrow_{\breducename}}
\newcommand\bmultistepname{\rightarrow^\star_{\breducename}}

\newcommand\bmultistep[3]{#1\vdash #2\;\bmultistepname\;#3}
\newcommand\bstdstep[3]{#1\vdash #2\;{\longmapsto_{\breducename}}\;#3}
\newcommand\bstdmultistep[3]{#1\vdash #2\;{\longmapsto^\star_{\breducename}}\;#3}

\newcommand\breduce[3]{#1 \vdash {#2}\;\breducename\; {#3}}
\newcommand\errreduce[3]{#1 \vdash {#2}\;\errreducename\; {#3}}
\newcommand\propreduce[3]{#1 \vdash {#2}\;\propreducename\; {#3}}
\newcommand\bvreduce[3]{#1 \vdash {#2}\;\bvreducename\; {#3}}
\newcommand\bstep[3]{#1 \vdash {#2}\;\rightarrow_{\breducename}\; {#3}}
\newcommand\bclosedstep[2]{{#1}\;\rightarrow_{\breducename}\; {#2}}

\newcommand\laxparstep{\rightrightarrows_\mathbf{a}}
\newcommand\maxparstep{\rightrightarrows'_\mathbf{a}}


\newcommand{\xdown}[1]{X_{#1 \downarrow}}
\newcommand{\xup}[1]{X_{#1 \uparrow}}
\newcommand{\xdownk}[2]{X_{#1 \downarrow #2}}
\newcommand{\xupk}[2]{X_{#1 \uparrow #2}}
\newcommand{\xblock}[1]{[x_{#1 1}, x_{#1 2} , \dots, x_{#1 \frac{1}{c}}]^{\top}}
\newcommand{\conpr}[2]{Pr[#1\,|\,#2]}
\newcommand{\priv}{{\bf priv}(X)}
\newcommand{\alt}[1]{{\bf alt}(X_{#1})}
\newcommand{\xbot}[1]{x_{#1 \frac{1}{c}}}
\newcommand{\cwp}[1]{(\epsilon, \delta, \Delta_{#1}, \Gamma)}


\newcommand\Arith{\mathcal{A}}
\newcommand\Barith{\mathcal{B}}

\newcommand\Var{\mathit{Var}}

\newcommand{\mvar}{x}
\newcommand\s[1]{\mathit{#1}}

\title{Streaming CW-Pivacy}
\date{\today}
\begin{document}
\maketitle
{\bf Problem Setting} \\
Let $X$ be a database in the streaming setting. Let $X_{i}$ represent the portion of $X$ that is currently held at time step $i$. We assume that at each time step, a fraction of $c$ of the database is replaced. We assume the oldest rows are always the ones replaced, and that $X$ has row drown i.i.d. from some distribution $D$. Let $n$ be the size of each $X_{i}$. This means that the first $n$ rows of $X$ constitute $X_{1}$, rows $cn+1$ through $cn+n$ constitute $X_{2}$, and so forth. $X$ has size $n+cn(t-1)$, where $t$ is the total number of time steps being considered. $1/c$ is the total number of time steps a given row will be present for.

We now consider a query $F: \, {\cal U}^{n} \rightarrow \mathbb{R}^{d}$ on each $X_{i}$. 
Let $D_{F}$ be the distribution that draws a database of size $n$, with each row chosen i.i.d. from $D$. Let $aux_{F}$ be $F$'s auxiliary information, which consists of the first (or last) $(1-c)n$ rows of the database. Now, assume $F$ is $(\epsilon, \delta, \Delta_{F}, \Gamma)$-CW private with some simulator $sim_{F}$, where $\Delta$ chooses the database according to $D_{F}$ and the auxiliary information is $aux_{F}$ as stated above.

Let $G(X) = (F(X_{1}), F(X_{2}) , \dots, F(X_{t}))$ be the composite query that runs $F$ at each time step. We show $G$ is $(t\epsilon, t \delta, \Delta, \Gamma)$ -CW private.
\\
\\
{\bf Notations }\\
For each $X_i$ of size $n$, we represent it by $1/c$ blocks (each has $cn$ rows). Namely, let $X_{i}=\xblock{i}$, where $x_{ij}$ denote the $j$th block in $X_{i}$. Let $\xdownk{i}{k} = [x_{i (k+1)}, x_{i (k+2)} , \dots, x_{i \frac{1}{c}}]^{\top}$ and $\xupk{i}{k} = [x_{i 1}, x_{i2} , \dots, x_{i (\frac{1}{c}-k)}]^{\top}$. We use $\xdown{i}$ to denote $\xdownk{i}{1}$ and $\xup{i}$ to denote $\xupk{i}{1}$. Notice that $\xdownk{i}{k} = \xupk{i+k}{k}$ (specifically $\xdown{i} = \xup{i+1}$), which are the shared blocks between $X_{i}$ and $X_{i+k}$.

Let $S=(S_{1}, S_{2}, \dots, S_{t} )$ be any set from $\mathbb {R} ^{d \times t}$, where $S_{j}$ is determined by the values of $(s_{1}, s_{2} , \dots, s_{j-1})$. Let $S_{-t}$ denote set $(S_{1}, S_{2}, \dots, S_{t-1} )$.
\\
\\
{\bf Proof}\\
We prove it inductively. 
\begin{enumerate}
\item The base case is when $G(X) = (F(X_{1}) , F(X_{2}))$. For any set $S= (S_{1}, S_{2})$, we have
\[
\conpr{G(X) \in S}{ \priv = v }
\]
$\priv =v$ will be omitted from now on.
\[
=\conpr{F(X_{2}) \in S{2}}{ F(X_{1}) \in S_{1}} \cdot Pr[F(X_{1}) \in S_{1}]
\]
\begin{equation}
=(\Sigma_{s_{1} \in S_{1}} \conpr{F(X_{2}) \in S{2}}{  F(X_{1}) = s_{1}} \cdot Pr[F(X_{1}) = s_{1}])\cdot Pr[F(X_{1}) \in S_{1}]
\end{equation}
We focus on $\conpr{F(X_{2}) \in S{2}}{  F(X_{1}) = s_{1}} $.
\[
\conpr{F(X_{2}) \in S{2}}{  F(X_{1}) = s_{1}} = \Sigma_{z} \conpr{F(X_{2}) \in S{2}}{  F(X_{1}) = s_{1}, \xup{2} =z} \cdot Pr [\xup{2} = z]
\]
\[
= \Sigma_{z} \conpr{F(\xup{2}, \xbot{2}) \in S{2}}{ F(x_{11}, \xdown{1})=s_{1}, \xup{2}=z} \cdot Pr [\xup{2} = z]
\]
\[
= \Sigma_{z} \conpr{F(\xup{2}, \xbot{2}) \in S{2}}{ F(x_{11}, \xup{2})=s_{1}, \xup{2}=z} \cdot Pr [\xup{2} = z]
\]
Given $\xup{2}=z$ and $\xbot{2}$ and $x_{11}$ are i.i.d. generated, functions $F(\xup{2}, \xbot{2})$ and $F(x_{11}, \xup{2})$ are independent with each other. Only $S_{2}$ depends on the value $s_{1}$. By the assumption that $F$ is $\cwp{F}$-CW private with simulator $sim_{F}$, we have
\[
\leq \Sigma_{z} (e^{\epsilon}  \conpr{sim_{F}(\alt{2}) \in S{2}}{ F(X_{1})=s_{1}, \xup{2}=z}+\delta) \cdot Pr [\xup{2} = z]
\]
\[
=e^{\epsilon}  \conpr{sim_{F}(\alt{2}) \in S{2}}{ F(X_{1})=s_{1}}+\delta
\]
By equation (1), we have
\[
Pr [G(X) \in S] \leq e^{\epsilon}  Pr({sim_{F}(\alt{2}) \in S{2}}, { F(X_{1}) \in S_{1}})+\delta
\]
\begin{equation}
=e^{\epsilon} \conpr{ F(X_{1}) \in S_{1}} {{sim_{F}(\alt{2}) \in S{2}}} \cdot Pr [{sim_{F}(\alt{2}) \in S{2}}] +\delta
\end{equation}
Similarly, $
\conpr{ F(X_{1}) \in S_{1}} {sim_{F}(\alt{2}) \in S{2}}=
$
\begin{equation}
\Sigma_{s_{2}\in S_{2}} \conpr{ F(X_{1}) \in S_{1}} {sim_{F}(\alt{2}) = s_{2}} \cdot Pr [sim_{F}(\alt{2}) = s_{2}]
\end{equation}
We focus on $\conpr{ F(X_{1}) \in S_{1}} {sim_{F}(\alt{2}) = s_{2}}$, which equals to
\[
\Sigma_{z} \conpr{ F(X_{1}) \in S_{1}} {sim_{F}(\alt{2}) = s_{2}, \xdown{1} =z} \cdot Pr[\xdown{1}=z]
\]
\[
=\Sigma_{z} \conpr{ F(x_{11}, \xup{2}) \in S_{1}} {sim_{F}({\bf alt}(\xdown{1}, \xbot{2})) = s_{2}, \xdown{1} =z} \cdot Pr[\xdown{1}=z]
\]
Notice that $sim_{F}({\bf alt}\xdown{1}, \xbot{2}))$ can be seen a composite function $sim_{F} \circ {\bf alt}$ on $\xbot{2}$. Given $\xdown{1}=z$ and  that $\xbot{2}$ and $x_{11}$ are i.i.d. generated, functions $F(x_{11}, \xdown{1})$ and $sim_{F}({\bf alt}(\xdown{1}, \xbot{2}))$ are independent with each other. Only $S_{1}$ depends on the value $s_{2}$. By the assumption that $F$ is $\cwp{F}$-CW private with simulator $sim_{F}$, we have
\[
\leq (e^{\epsilon} \Sigma_{z} \conpr{ sim_{F}(X_{1}) \in S_{1}} {sim_{F}(\alt{2}) = s_{2},\xdown{1} =z}+\delta) \cdot Pr[\xdown{1}=z]
\]
\[
=e^{\epsilon} \conpr{ sim_{F}(X_{1}) \in S_{1}} {sim_{F}(\alt{2}) = s_{2}}+\delta
\]
By equation (2) and (3), we have
\[
\conpr{ F(X_{1}) \in S_{1}} {sim_{F}(\alt{2}) \in S{2}} \leq e^{\epsilon} \conpr{ sim_{F}(X_{1}) \in S_{1}} {sim_{F}(\alt{2}) \in S_{2}}+\delta
\]
and therefore
\[
Pr[G(X) \in S] \leq e^{2\epsilon} Pr[sim_{F} (\alt{1}) \in S_{1} , sim_{F} (\alt{2}) \in S_{2}] + 2\delta
\]
\item Assume it is true for $G_{t-1}(X)=(F(X_{1}) , \dots, F(X_{t-1}))$, we prove it holds for $G_{t}(X)$. For any set $S={S_{1}, S_{2}, \dots, S_{t}}$, we have
\begin{equation}
Pr[G_{t}(X) \in S] = \conpr{F(X_{t}) \in S_{t}}{ G_{t-1}(X) \in S_{-t}} \cdot Pr[ G_{t-1}(X) \in S_{-t}]
\end{equation}
By inductive assumption, we have
\[
Pr[ G_{t-1}(X) \in S_{-t}] \leq 
\]
\[
e^{(t-1) \epsilon} Pr [(sim_{F}(\alt{1}) , \dots , sim_{F}(\alt{t-1})) \in S_{-t}] + (t-1)\delta
\]
We focus on $\conpr{F(X_{t}) \in S_{t}}{ G_{t-1}(X) \in S_{-t}}$, which equals to
\[
=\Sigma_{z} \conpr{F(X_{t}) \in S_{t}}{ G_{t-1}(X) \in S_{-t} , \xup{t}=z}\cdot Pr[\xup{t}=z]
\]
For any database $X_{i}$ where $1 \leq i \leq t-1$, that shares no common block with $X_{t}$, $F(X_{t})$ is independent with $F(X_{i})$. For the simplicity of analysis, we assume every $X_{i}$ has shared at least one common block with $X_{t}$. 
\[
\conpr{F(X_{t}) \in S_{t}}{ G_{t-1}(X) \in S_{-t} , \xup{t}=z} =
\]
\[
\conpr{F(\xup{t}, \xbot{t}) \in S_{t}}{F(x_{11},\dots, x_{1(t-1), \xdownk{1}{t-1}}), \dots, F(x_{(t-1)1}, \xdown{t-1}) \in S_{-t}, \xup{t}=z}
\]
Given $\xup{t}=z$, each $F(X_{i})$ can be seen as a function on $\xupk{i}{t-i}=[x_{i1},\dots , x_{i(t-i)}]^{\top}$. Notice $x_{ij}=x_{i'j'}$ as long as $i+j=i'+j'$. Given $\xup{t}=z$, $G_{t-1}(X)$ is a function on $\xupk{1}{t-1}=[x_{11},\dots , x_{1(t-1)}]^{\top}$. Since $\xbot{t}$ is independent with each block in $\xupk{1}{t-1}$, functions $F(\xup{t}, \xbot{t})$ and $G_{t-1}(X)$ are independent with each other, given $\xup{t}=z$. Only $S_{t}$ depends on the value of $G_{t-1}(X)$. By the assumption that $F$ is $\cwp{F}$-CW private with simulator $sim_{F}$, we have
\[
\leq \Sigma_{z} (e^{\epsilon} \conpr{sim_{F} (\alt{t}) \in S_{t}}{G_{t-1}(X) \in S_{-t},  \xup{t}=z} + \delta)\cdot Pr[\xup{t}=z]
\]
\[
=e^{\epsilon} \conpr{sim_{F} (\alt{t}) \in S_{t}}{G_{t-1}(X) \in S_{-t}} + \delta
\]
By equation (4), we have
\begin{equation}
Pr[G_{t}(X) \in S] \leq e^{\epsilon} Pr[ sim_{F} (\alt{t}) \in S_{t}, G_{t-1}(X) \in S_{-t}] + \delta
\end{equation}
We now focus on $Pr[ sim_{F} (\alt{t}) \in S_{t}, G_{t-1}(X) \in S_{-t}] $, which equals to
\[
=\conpr{F(X_{t-1}) \in S_{t-1}}{sim_{F} (\alt{t}) \in S_{t}, G_{t-2}(X) \in S_{-(t-1)}}
\]
\[
\cdot Pr [sim_{F} (\alt{t}) \in S_{t}, G_{t-2}(X) \in S_{-(t-1)}]
\]
Similarly, we can show
\[
\conpr{F(X_{t-1}) \in S_{t-1}}{sim_{F} (\alt{t}) \in S_{t}, G_{t-2}(X) \in S_{-(t-1)}}
\]
\[
\leq e^{\epsilon} \conpr{sim_{F}(\alt{t-1}) \in S_{t-1}}{sim_{F} (\alt{t}) \in S_{t}, G_{t-2}(X) \in S_{-(t-1)}}+\delta
\]
\end{enumerate}
\end{document}
